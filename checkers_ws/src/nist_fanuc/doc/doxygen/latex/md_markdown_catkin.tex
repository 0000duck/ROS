For a good background read the doc at htt\-: {\ttfamily \href{http://catkin-tools.readthedocs.org/en/latest/}{\tt http\-://catkin-\/tools.\-readthedocs.\-org/en/latest/}}

Quick overview of useful catkin commands\-: \begin{DoxyVerb}catkin clean --build --devel
\end{DoxyVerb}


just clean a single package\-: \begin{DoxyVerb}    catkin clean PKGNAME
\end{DoxyVerb}


Building ros packages \begin{DoxyVerb}catkin build   
\end{DoxyVerb}


Setting and unsetting C\-Make options\-: \begin{DoxyVerb}    catkin config --cmake-args -DENABLE_CORBA=ON -DCORBA_IMPLEMENTATION=OMNIORB
    catkin config --no-cmake-args
\end{DoxyVerb}


Create “\-Debug” and “\-Release” profiles and then build them in independent build and devel spaces\-: \begin{DoxyVerb}catkin config --profile debug -x _debug --cmake-args -DCMAKE_BUILD_TYPE=Debug
catkin config --profile release -x _release --cmake-args -DCMAKE_BUILD_TYPE=Release
catkin build --profile debug
catkin build --profile release
\end{DoxyVerb}


build specific packages (however this will build dependency packages) \begin{DoxyVerb}build PKG [PKG ...]
\end{DoxyVerb}


\subsection*{Initialize catkin workspace }

\begin{DoxyVerb}source /opt/ros/indigo/setup.bash          # Source ROS indigo to use Catkin
mkdir -p /tmp/quickstart_ws/src            # Make a new workspace and source space
cd /tmp/quickstart_ws                      # Navigate to the workspace root
catkin init     
\end{DoxyVerb}


\subsection*{Build one catkin project }

If you’re only interested in building a single package in a workspace, you can also use the --no-\/deps option along with a package name. This will skip all of the package’s dependencies, build the given package, and then exit. \begin{DoxyVerb}catkin build PKG --no-deps # Build PKG only
\end{DoxyVerb}


\subsection*{Use some but not all of the processor cores }

You can control the number of build jobs. Typically a job controller is used and all the cores are assigned make files, which halts G\-U\-I interaction.

You can control the maximum number of packages allowed to build in parallel by using the -\/p or --parallel-\/packages option and you can change the number of make jobs available with the -\/j or --jobs option.

To disable the job server, you can use the --no-\/jobserver option. \begin{DoxyVerb}catkin build PKG --no-deps -p 7 # Build PKG only, and use only 7 or 8 cores
\end{DoxyVerb}


\subsection*{Building With Warnings }

It can sometimes be useful to compile with additional warnings enabled across your whole catkin workspace. To achieve this, use a command similar to this\-: \begin{DoxyVerb}$ catkin build -v --cmake-args -DCMAKE_C_FLAGS="-Wall -W -Wno-unused-parameter"
\end{DoxyVerb}


This command passes the {\ttfamily -\/\-D\-C\-M\-A\-K\-E\-\_\-\-C\-\_\-\-F\-L\-A\-G\-S=}... argument to all invocations of cmake. 